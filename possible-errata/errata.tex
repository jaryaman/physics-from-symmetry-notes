% Preamble ==================================================================
\documentclass[11pt]{article}
\usepackage{geometry}
\geometry{verbose,tmargin=2.5cm,bottom= 1.5cm,lmargin=2.5cm,rmargin=2.5cm}
\usepackage{float}
\usepackage{graphicx}
\usepackage{amsmath}
\usepackage{amssymb}
\usepackage{enumitem}
\usepackage{mathtools}

\usepackage{tensor}
\usepackage{cancel}
\usepackage{wasysym}
\usepackage{braket}

\usepackage{amsthm} % theorem

\numberwithin{equation}{section}

\usepackage{titlesec,dsfont}

%Format section heading style
\usepackage{sectsty}
\sectionfont{\sffamily\bfseries\large}
\subsectionfont{\sffamily\normalsize\slshape}
\subsubsectionfont{\sffamily\small\itshape}
\paragraphfont{\sffamily\small\textbf}


%Put period after section number
\makeatletter
\def\@seccntformat#1{\csname the#1\endcsname.\quad}
\makeatother

%Bibliography


%Format captions
\usepackage[ labelsep=period, justification=raggedright, margin=10pt,font={small},labelfont={small,normal,bf,sf}]{caption}

\setlength{\parskip}{0ex} %No space between paragraphs.

\renewcommand{\familydefault}{\sfdefault}

\newcommand\indep{\protect\mathpalette{\protect\independenT}{\perp}}
\newcommand{\nindep}{\not\!\perp\!\!\!\perp}
\def\independenT#1#2{\mathrel{\rlap{$#1#2$}\mkern2mu{#1#2}}}

%PUT ME LAST--------------------------------------------------
\usepackage[colorlinks=true
,urlcolor=blue
,anchorcolor=blue
,citecolor=blue
,filecolor=blue
,linkcolor=black
,menucolor=blue
,linktocpage=true
,pdfproducer=medialab
,pdfa=true
]{hyperref}

\makeatother %Put this last of all

% Symbol definitions
\newcommand{\defeq}{\coloneqq}
\renewcommand{\d}[1]{\ensuremath{\operatorname{d}\!{#1}}}
\newcommand{\deriv}[2]{\frac{\ensuremath{\operatorname{d}\!{#1}}}{\ensuremath{\operatorname{d}\!{#2}}}}
\newcommand{\derivn}[3]{\frac{\ensuremath{\operatorname{d}^{#1}\!{#2}}}{\ensuremath{\operatorname{d}\!{#3}^{#1}}}}
\DeclareMathOperator{\diag}{diag}
\DeclareMathOperator{\tr}{tr}
\newcommand{\tn}[2]{\tensor{#1}{#2}}

% Make theorems bold
\makeatletter
\def\th@plain{%
  \thm@notefont{}% same as heading font
  \itshape % body font
}
\def\th@definition{%
  \thm@notefont{}% same as heading font
  \normalfont % body font
}
\makeatother

% Theorem definitions
\newtheorem{thm}{Theorem}[section]
\newtheorem{defn}{Definition}[section]
\newtheorem{cor}{Corollary}[section]
\newtheorem{prop}{Property}[section]
\newtheorem{rle}{Rule}[section]
\newtheorem{lma}{Lemma}[section]

\newcommand{\bs}{\boldsymbol}

%Preamble end--------------------------------------------------


\begin{document}



\begin{flushleft}
\textbf{\Large Possible errata in Physics from Symmetry (Second Edition)}
\end{flushleft}

\begin{flushleft}
Author: Juvid Aryaman

Last compiled: \today
\end{flushleft}


\section{$(\frac{1}{2}, \frac{1}{2})$ Representation (p82)}

On p82, Eq.(3.225), we have

\begin{equation}
v \rightarrow v' = \tn{{v'}}{_a_{\dot{b}}} = \tn{{\left(e^{i \vec{\theta} \frac{\vec{\sigma}}{2} + \vec{\phi} \frac{\vec{\sigma}}{2}} \right)}}{_a^c} \tn{v}{_c_{\dot{d}}} \left(\tn{{\left(e^{-i \vec{\theta} \frac{\vec{\sigma^*}}{2} + \vec{\phi} \frac{\vec{\sigma^*}}{2}} \right)}}{^{\dot{d}}_{\dot{b}}}\right)^T  \nonumber
\end{equation}

\subsection{Suggested correction}

I suggest this equation should read
\begin{equation}
v \rightarrow v' = \tn{{v'}}{_a_{\dot{b}}} = \tn{{\left(e^{i \vec{\theta} \frac{\vec{\sigma}}{2} + \vec{\phi} \frac{\vec{\sigma}}{2}} \right)}}{_a^c} \tn{v}{_c_{\dot{d}}} \tn{{\left(e^{i \vec{\theta} \frac{\vec{\sigma}}{2} - \vec{\phi} \frac{\vec{\sigma}}{2}} \right)}}{^{\dot{d}}_{\dot{b}}}  \label{eq:my-fix}
\end{equation}
If Eq.(3.225) is indeed in error, then the error appears to propagate through the rest of Section 3.7.8.

\subsection{Rationale}
We found in the previous section that dotted and undotted indices transform differently, namely that
\begin{align}
\tn{{\chi'}}{^{\dot{a}}} &= \tn{\Lambda}{^{\dot{a}}_{\dot{b}}} \tn{\chi}{^{\dot{b}}} \\
\tn{\Lambda}{^{\dot{a}}_{\dot{b}}} &= \tn{{\left(e^{i \vec{\theta} \frac{\vec{\sigma}}{2} - \vec{\phi}\frac{\vec{\sigma}}{2}} \right)}}{^{\dot{a}}_{\dot{b}}} \label{eq:right-chiral}
\end{align}
(see Eq.(3.219) and Eq.(3.222) from Physics from Symmetry), and also
\begin{align}
\tn{{\chi'}}{_a} &= \tn{\Lambda}{_a^b} \tn{\chi}{_b} \\
\tn{\Lambda}{_a^b} &= \tn{{\left(e^{i \vec{\theta} \frac{\vec{\sigma}}{2} + \vec{\phi}\frac{\vec{\sigma}}{2}} \right)}}{_a^b} \label{eq:left-chiral}
\end{align}
(see Eq.(3.220) and Eq.(3.221) from Physics from Symmetry).

I assume that $v'$ may be written explicitly as follows:
\begin{equation}
\tn{{v'}}{_a_{\dot{b}}} = \tn{\Lambda}{_a^c} \tn{v}{_c_{\dot{d}}} \tn{\Lambda}{^{\dot{d}}_{\dot{b}}} \label{eq:my-starting-point}
\end{equation}
Substituting Eq.\eqref{eq:left-chiral} and Eq.\eqref{eq:right-chiral} into the above equation trivially yields Eq.\eqref{eq:my-fix}.

I believe Eq.\eqref{eq:my-starting-point} and Eq.(3.225) from Physics from Symmetry are incompatible. There is no way to derive the complex conjugate $e^{+\vec{\phi} \frac{\vec{\sigma^*}}{2}}$ from $e^{- \vec{\phi}\frac{\vec{\sigma}}{2}}$ in the (dotted) right-chiral terms of the transformation.

\section{Spinors and Charge Conjugation (p86)}

On p86, Eq.(3.243) reads
\begin{equation}
\tilde{\Psi} \rightarrow \tilde{\Psi}' = \begin{pmatrix}
e^{-\frac{\vec{\theta}}{2} \vec{\sigma}} & 0 \\
0 & e^{\frac{\vec{\theta}}{2} \vec{\sigma}}
\end{pmatrix} \begin{pmatrix}
\chi_L \\
\chi_R
\end{pmatrix}
\end{equation}

\subsection{Suggested correction}
I suggest this equation should read
\begin{equation}
\tilde{\Psi} \rightarrow \tilde{\Psi}' = \begin{pmatrix}
e^{-\frac{\vec{\theta}}{2} \vec{\sigma}} & 0 \\
0 & e^{\frac{\vec{\theta}}{2} \vec{\sigma}}
\end{pmatrix} \begin{pmatrix}
\chi_R \\
\chi_L
\end{pmatrix}
\end{equation}

\subsection{Rationale}
$\tilde{\Psi}$ is defined as an object where the chirality of each component is naively flipped Eq.(3.241)
\begin{equation}
\tilde{\Psi} = \begin{pmatrix}
\chi_R \\
\chi_L
\end{pmatrix}.
\end{equation}
\noindent Therefore, a Lorentz boost should operate on the same initial object. My suggestion for $\tilde{\Psi}$ would indeed not transform like a Dirac spinor under $\Lambda_{(\frac{1}{2},0) \oplus (0, \frac{1}{2})}$; it would transform under $\Lambda_{(0, \frac{1}{2},0) \oplus (\frac{1}{2}, 0)}$.

\newpage

\end{document}