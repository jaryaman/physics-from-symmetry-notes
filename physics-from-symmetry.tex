% Preamble ==================================================================
\documentclass[11pt]{article}
\usepackage{geometry}
\geometry{verbose,tmargin=2.5cm,bottom= 1.5cm,lmargin=2.5cm,rmargin=2.5cm}
\usepackage{float}
\usepackage{graphicx}
\usepackage{amsmath}
\usepackage{amssymb}
\usepackage{enumitem}
\usepackage{mathtools}

\usepackage{amsthm} % theorem

\numberwithin{equation}{section}

\usepackage{titlesec,dsfont}

%Format section heading style
\usepackage{sectsty}
\sectionfont{\sffamily\bfseries\large}
\subsectionfont{\sffamily\normalsize\slshape}
\subsubsectionfont{\sffamily\small\itshape}
\paragraphfont{\sffamily\small\textbf}


%Put period after section number
\makeatletter
\def\@seccntformat#1{\csname the#1\endcsname.\quad}
\makeatother

%Bibliography
\usepackage[round]{natbib}
\bibliographystyle{genetics}

%Format captions
\usepackage[ labelsep=period, justification=raggedright, margin=10pt,font={small},labelfont={small,normal,bf,sf}]{caption}

\setlength{\parskip}{0ex} %No space between paragraphs.

\renewcommand{\familydefault}{\sfdefault}

\newcommand\indep{\protect\mathpalette{\protect\independenT}{\perp}}
\newcommand{\nindep}{\not\!\perp\!\!\!\perp}
\def\independenT#1#2{\mathrel{\rlap{$#1#2$}\mkern2mu{#1#2}}}

%PUT ME LAST--------------------------------------------------
\usepackage[colorlinks=true
,urlcolor=blue
,anchorcolor=blue
,citecolor=blue
,filecolor=blue
,linkcolor=black
,menucolor=blue
,linktocpage=true
,pdfproducer=medialab
,pdfa=true
]{hyperref}

\makeatother %Put this last of all

% Symbol definitions
\newcommand{\defeq}{\coloneqq}
\renewcommand{\d}[1]{\ensuremath{\operatorname{d}\!{#1}}}

% Make theorems bold
\makeatletter
\def\th@plain{%
  \thm@notefont{}% same as heading font
  \itshape % body font
}
\def\th@definition{%
  \thm@notefont{}% same as heading font
  \normalfont % body font
}
\makeatother

% Theorem definitions
\newtheorem{thm}{Theorem}[section]
\newtheorem{defn}{Definition}[section]
\newtheorem{cor}{Corollary}[section]
\newtheorem{prop}{Property}[section]
\newtheorem{rle}{Rule}[section]
\newtheorem{lma}{Lemma}[section]

%Preamble end--------------------------------------------------


\begin{document}



\begin{flushleft}
\textbf{\Large Notes on Physics from Symmetry}
\end{flushleft}

\begin{flushleft}
Author: Juvid Aryaman

Last compiled: \today
\end{flushleft}

\noindent This document contains my personal notes on Jakob Schwichtenberg's Physics from Symmetry \citep{Schwichtenberg15}.


\section{Special relativity}

\subsection{Definitions and postulates}

In special relativity, \textbf{inertial frames of reference} are coordinate systems moving with constant velocity relative to each other. Special relativity has two basic postulates:
\begin{enumerate}
\item \textbf{The principal of relativity}: The laws of physics are the same in all inertial frames of reference.
\item \textbf{The invariance of the speed of light}: The velocity of light has the same value $c$ in all inertial frames of reference.
\end{enumerate}

\begin{thm}[Invariant of special relativity]\label{thm:invariant-sr}
Consider two events $A$ and $B$ in an inertial observer $O$'s frame of reference. Let the time interval measured by $O$ between the two events be $(\Delta t)$, and the three spatial intervals be $(\Delta x)$, $(\Delta y)$, $(\Delta z)$. Then, the quantity
\begin{equation}
(\Delta s)^2 \defeq (\Delta c t)^2 - (\Delta x)^2 - (\Delta y)^2 - (\Delta y)^2
\end{equation}
is \textbf{invariant} between all frames of reference. I.e.
\begin{equation}
(\Delta s') = (\Delta s)
\end{equation}
for any inertial frame of reference $O'$.
\end{thm}
\noindent Theorem \ref{thm:invariant-sr} follows directly from the invariance of the speed of light.

\begin{defn}[Proper time] \label{defn:proper-time}
Proper time, $\tau$, is the time measured by an observer in the special frame of reference where the object in question is at rest. In this frame of reference,
\begin{equation}
(\Delta s)^2 = (c \Delta \tau)^2.
\end{equation}
In the infinitesimal limit
\begin{equation}
(\d s)^2 = (c \d \tau)^2.
\end{equation}
\end{defn}
Physically, Defn.~\ref{defn:proper-time} means that all observers agree on the time interval between events for an observer who travels with the object in question. However, different observers \textbf{do not} in general agree on the time interval between events generally: $(\Delta t) \neq (\Delta t')$ -- this is called \textbf{time dilation}. 

\subsection{$c$ is an upper speed limit}
All observers agree on the value of $(\d s)^2=(c \d \tau)^2$. Furthermore, we commonly assume that there exists a minimal proper time of $\tau = 0$ for two events if $\Delta s^2 = 0$. We can therefore write that when $\tau = 0$
\begin{equation}
c^2 = \frac{(\d x)^2 + (\d y)^2 + (\d z)^2}{(\d t)^2}
\end{equation}
between two events with an infinitesimal distance. We can equate the right-hand side with a squared velocity, and hence
\begin{equation}
\tau = 0 \implies c^2 = v^2\ 
\end{equation}
so
\begin{equation}
(\d s)^2 \geq 0 \implies c^2 \geq v^2
\end{equation}
for \textbf{any} pair of events (which are causally connected, although how this follows is not immediately clear to me right now). 

\newpage
\bibliography{physics-from-symmetry.bib} 

\end{document}