% Preamble ==================================================================
\documentclass[11pt]{article}
\usepackage{geometry}
\geometry{verbose,tmargin=2.5cm,bottom= 1.5cm,lmargin=2.5cm,rmargin=2.5cm}
\usepackage{float}
\usepackage{graphicx}
\usepackage{amsmath}
\usepackage{amssymb}
\usepackage{enumitem}
\usepackage{mathtools}

\usepackage{tensor}
\usepackage{cancel}
\usepackage{wasysym}

\usepackage{amsthm} % theorem

\numberwithin{equation}{section}

\usepackage{titlesec,dsfont}

%Format section heading style
\usepackage{sectsty}
\sectionfont{\sffamily\bfseries\large}
\subsectionfont{\sffamily\normalsize\slshape}
\subsubsectionfont{\sffamily\small\itshape}
\paragraphfont{\sffamily\small\textbf}


%Put period after section number
\makeatletter
\def\@seccntformat#1{\csname the#1\endcsname.\quad}
\makeatother

%Bibliography
\usepackage[round]{natbib}
\bibliographystyle{genetics}

%Format captions
\usepackage[ labelsep=period, justification=raggedright, margin=10pt,font={small},labelfont={small,normal,bf,sf}]{caption}

\setlength{\parskip}{0ex} %No space between paragraphs.

\renewcommand{\familydefault}{\sfdefault}

\newcommand\indep{\protect\mathpalette{\protect\independenT}{\perp}}
\newcommand{\nindep}{\not\!\perp\!\!\!\perp}
\def\independenT#1#2{\mathrel{\rlap{$#1#2$}\mkern2mu{#1#2}}}

%PUT ME LAST--------------------------------------------------
\usepackage[colorlinks=true
,urlcolor=blue
,anchorcolor=blue
,citecolor=blue
,filecolor=blue
,linkcolor=black
,menucolor=blue
,linktocpage=true
,pdfproducer=medialab
,pdfa=true
]{hyperref}

\makeatother %Put this last of all

% Symbol definitions
\newcommand{\defeq}{\coloneqq}
\renewcommand{\d}[1]{\ensuremath{\operatorname{d}\!{#1}}}
\DeclareMathOperator{\diag}{diag}

% Make theorems bold
\makeatletter
\def\th@plain{%
  \thm@notefont{}% same as heading font
  \itshape % body font
}
\def\th@definition{%
  \thm@notefont{}% same as heading font
  \normalfont % body font
}
\makeatother

% Theorem definitions
\newtheorem{thm}{Theorem}[section]
\newtheorem{defn}{Definition}[section]
\newtheorem{cor}{Corollary}[section]
\newtheorem{prop}{Property}[section]
\newtheorem{rle}{Rule}[section]
\newtheorem{lma}{Lemma}[section]

%Preamble end--------------------------------------------------


\begin{document}



\begin{flushleft}
\textbf{\Large Notes on Physics from Symmetry}
\end{flushleft}

\begin{flushleft}
Author: Juvid Aryaman

Last compiled: \today
\end{flushleft}

\noindent This document contains my personal notes on Jakob Schwichtenberg's Physics from Symmetry \citep{Schwichtenberg15}, with a sprinkling of notes from my undergraduate physics course in quantum field theory (and, to a lesser extent, general relativity).


\section{Special relativity}

\subsection{Definitions and postulates} \label{sec:postualtes-sr}

In special relativity, \textbf{inertial frames of reference} are coordinate systems moving with constant velocity relative to each other. Special relativity has two basic postulates:
\begin{enumerate}
\item \textbf{The principal of relativity}: The laws of physics are the same in all inertial frames of reference.
\item \textbf{The invariance of the speed of light}: The velocity of light has the same value $c$ in all inertial frames of reference.
\end{enumerate}

\begin{thm}[Invariant of special relativity]\label{thm:invariant-sr}
Consider two events $A$ and $B$ in an inertial observer $O$'s frame of reference. Let the time interval measured by $O$ between the two events be $(\Delta t)$, and the three spatial intervals be $(\Delta x)$, $(\Delta y)$, $(\Delta z)$. Then, the quantity
\begin{equation}
(\Delta s)^2 \defeq (\Delta c t)^2 - (\Delta x)^2 - (\Delta y)^2 - (\Delta y)^2
\end{equation}
is \textbf{invariant} between all frames of reference. I.e.
\begin{equation}
(\Delta s') = (\Delta s)
\end{equation}
for any inertial frame of reference $O'$.
\end{thm}
\noindent Theorem \ref{thm:invariant-sr} follows directly from the invariance of the speed of light (consider a pair of mirrors, for two observers with relative velocity).

\begin{defn}[Proper time] \label{defn:proper-time}
Proper time, $\tau$, is the time measured by an observer in the special frame of reference where the object in question is at rest. In this frame of reference,
\begin{equation}
(\Delta s)^2 = (c \Delta \tau)^2.
\end{equation}
In the infinitesimal limit
\begin{equation}
(\d s)^2 = (c \d \tau)^2.
\end{equation}
\end{defn}
Physically, Defn.~\ref{defn:proper-time} means that all observers agree on the time interval between events for an observer who travels with the object in question. However, different observers \textbf{do not} in general agree on the time interval between events generally: $(\Delta t) \neq (\Delta t')$ -- this is called \textbf{time dilation}. 

\subsection{$c$ is an upper speed limit}
All observers agree on the value of $(\d s)^2=(c \d \tau)^2$. Furthermore, we commonly assume that there exists a minimal proper time of $\tau = 0$ for two events if $\Delta s^2 = 0$. We can therefore write that when $\tau = 0$
\begin{equation}
c^2 = \frac{(\d x)^2 + (\d y)^2 + (\d z)^2}{(\d t)^2}
\end{equation}
between two events with an infinitesimal distance. We can equate the right-hand side with a squared velocity, and hence
\begin{equation}
\tau = 0 \implies c^2 = v^2\ 
\end{equation}
so
\begin{equation}
(\d s)^2 \geq 0 \implies c^2 \geq v^2
\end{equation}
for \textbf{any} pair of events (which are causally connected, although how this follows is not immediately clear to me right now). 

\subsection{Tensor notation and Minkowski spacetime}

\begin{defn}[Four-vector (contravariant)]
A position four-\textbf{vector} is defined as
\begin{equation}
x^\mu = \begin{pmatrix}ct\\x\\y\\z\end{pmatrix} \equiv \begin{pmatrix}x^0\\x^1\\x^2\\x^3\end{pmatrix}.
\end{equation}
\end{defn}
\begin{defn}[Minkowski metric]
The Minkowski metric is defined as 
\begin{equation}
\eta_{\mu \nu} = \diag(1,-1,-1,-1).
\end{equation}
$\eta$ is used to compute distances and lengths in Minkowski space.
\end{defn}
\noindent We define $\eta^{\mu \nu}$ through the relation
\begin{equation}
\eta^{\mu \nu} \eta_{\nu \sigma} = \tensor{\delta}{^\mu_\sigma}
\end{equation}
where we have appled the \textbf{Einstein summation convention}, where a repeated Greek index implies a summation from 0 to 3 (where the zeroth index is time), and a repeated Roman index is summed from 1 to 3. Hence, for a matrix multiplication between two 3$\times$3 matricies $A$ and $B$, $(AB)_{ij} = A_{ik}B_{kj}$, and $(A^T)_{ij}=A_{ji}$.


\begin{defn}[One-form (covariant vector)]
We define a one-form as 
\begin{equation}
x_\mu = \eta_{\mu \nu}x^\nu.
\end{equation}
\end{defn} 
\noindent Thus,
\begin{equation}
\d s^2 = \eta_{\mu \nu} \d x^\mu \d x^\nu.
\end{equation}
\begin{defn}[Scalar product]
A scalar product between four-vectors $x$ and $y$ is defined as
\begin{equation}
x \cdot y \defeq x^\mu y^\nu \eta_{\mu \nu} = x_\mu y_\nu \eta^{\mu \nu} = x^\mu y_\mu = x_\nu y^\nu
\end{equation}
due to the symmetry of the metric: $\eta_{\mu \nu} = \eta_{\nu \mu}$.
\end{defn}

\paragraph{Ordering (spacing) of indicies} In order to be able to freely raise/lower indicies (without repeatedly writing the metric tensor), we can impose an ordering upon indicies of tensor fields -- which we can represent typographically with spacing between tensor indicies. A metric $g_{ij}$ (or $g^{ij}$) has the effect of lowering (or raising) a repeated index. For example,
\begin{equation}
g_{iq} \tensor{T}{^{abcd}_{efgh}^{ijkl}_{mnop}}  = \tensor{T}{^{abcd}_{efghq}^{jkl}_{mnop}}.
\end{equation}
(Proof of this, I imagine, requires background in differential geometry?)

\subsection{Lorentz transformations}
From the invariant of SR (Theorem~\ref{thm:invariant-sr}), we have
\begin{equation}
\d s'^2 = \d x'_\mu \d x'_\nu \eta^{\mu \nu} = \d x_\mu \d x_\nu \eta^{\mu \nu}
\end{equation}
for all reference frames. We denote $\Lambda$ as a (1,1) tensor field, which transforms a four-vector from one reference frame to another:
\begin{equation}
\d x'^\mu = \Lambda^\mu{}_{\nu} \d x ^\nu
\end{equation}
which leaves the $\d s^2$ invariant, i.e. $\d s'^2 = ds^2$. It follows that 
\begin{align}
\eta_{\mu \nu} &= \tensor{\Lambda}{^\sigma_\mu} \tensor{\Lambda}{^\delta_\nu} \eta_{\sigma \delta} \label{eq:lorentz-tfm-characteristic}\\
\eta &= \Lambda^T \eta \Lambda. \nonumber
\end{align}
The physical meaning of Eq.\eqref{eq:lorentz-tfm-characteristic} is that Lorentz transformations leave the scalar product of Minkowski spacetime invariant: i.e. changes between frames of reference that respect the two postualtes of special relativity (Section~\ref{sec:postualtes-sr}). Conservation of the scalar product is analogous to rigid rotation ($O$) in Euclidean space ($a \cdot b = a' \cdot b' = a^T O^T O b \implies O^T I O = I$), which preserves orientation ($\det(\Lambda)=1$).


Note that $\tensor{\Lambda}{^\mu_\nu} \neq \tensor{\Lambda}{_\nu^\mu}$. Beginning with Eq.\eqref{eq:lorentz-tfm-characteristic},
\begin{align*}
\tensor{\Lambda}{^\mu_\rho} \tensor{\Lambda}{^\nu_\sigma} \eta_{\mu \nu} &= \eta_{\rho \sigma}
\end{align*}
we can raise one index, and lower one index, of $\tensor{\Lambda}{^\nu_\sigma}$
\begin{align}
\tensor{\Lambda}{^\mu_\rho} \eta_{\mu \nu} \tensor{\Lambda}{^\nu_\sigma} \eta_{\nu \mu} \eta^{\sigma \lambda}  &= \eta_{\rho \sigma} \eta_{\nu \mu} \eta^{\sigma \lambda} \nonumber \\
\tensor{\Lambda}{^\mu_\rho} \tensor{\Lambda}{_\mu^\lambda} \cancel{\eta_{\mu \nu}} &=  \cancel{\eta_{\mu \nu}} \tensor{\delta}{_\rho^\lambda} \nonumber \\
\tensor{\Lambda}{^\mu_\rho} \tensor{\Lambda}{_\mu^\lambda} &= \tensor{\delta}{_\rho^\lambda}
\end{align}
so we see that $\tensor{\Lambda}{_\nu^\mu}$ is the inverse of $\tensor{\Lambda}{^\mu_\nu}$.


\section{Lie group theory}
\subsection{Invariance, symmetry, and covariance}
We call a quantity \textbf{invariant} if it does not change under particular transformations. E.g. if we transform $A,B,C,... \rightarrow A',B',C',...$ and we have
\begin{equation}
F(A',B',C',...) = F(A,B,C,...)
\end{equation}
then we say $F$ is invariant under this transformation. \textbf{Symmetry} is defined as invariance under a transformation (or class of transformations). An equation is covariant if it takes the same form when objects in it are transformed. \textit{All physical laws must be covariant under Lorentz transformations.} 

Group theory describes the properties of particular sets of transformations: the invariances under such groups allows us to mathematically describe symmetry. For example, the set of rotations about the origin of a square by $n\pi/2$ form a \textbf{discrete group}, and leave the set of points which constitute the square invariant under the transformation. The set of rotations about the origin of a circle form a \textbf{continuous group}. We can use group theory to work with \textit{all} kinds of symmetries: symmetries which operate on vectors, equations, ...

\subsection{Groups}
\begin{defn}[Group axioms]
A group $(G, \circ)$ is a set $G$, together with a binary operation $\circ$ defined on $G$, that satisfies the following axioms
\begin{itemize}
\item Closure: For all $g_1, g_2 \in G$, $g_1 \circ g_2 \in G$
\item Identity element: There exists an identity element $e \in G$ such that for all $g \in G$, $g \circ e = g = e \circ g$
\item Inverse element: For each $g \in G$, there exists an inverse element $g^{-1} \in G$ such that $g \circ g^{-1} = e = g^{-1}g$.
\item Associativity: For all $g_1, g_2, g_3 \in G$, $g_1 \circ (g_2 \circ g_3) = (g_1 \circ g_2) \circ g_3$
\end{itemize}
\end{defn}

The set of all transformations that leave a given object invariant is called a \textbf{symmetry group}. For Minkowski spacetime, the object that is left invariant is the Minkowski metric, and the corresponding symmetry group is called the \textbf{Poincar\'{e} group}. Notice that the transformations which constitute a group are defined entirely independently from the object on which the transformations act.

\subsubsection{Rotations in two dimensions and $SO(2)$}
Consider the 2D rotation matrix 
\begin{equation}
R_\theta = \begin{pmatrix}
\cos(\theta)& -\sin(\theta)\\
\sin(\theta)& \cos(\theta)
\end{pmatrix} \label{eq:2d-rotation-matrix}
\end{equation}
and the two reflection matrices
\begin{equation}
P_x = \begin{pmatrix}
-1& 0\\
0& 1
\end{pmatrix} \qquad 
P_y = \begin{pmatrix}
1& 0\\
0& -1
\end{pmatrix}.
\end{equation}
These matrices satisfy the group axioms. We can uncover this group from a symmetry perspective. The above transformations leave the length of a vector unchanged, i.e.
\begin{equation}
a.a = a'.a'.
\end{equation}
Letting the transformation be represented by $a' = Oa$, it follows that all members of the group must satisfy
\begin{equation}
O^T O = I.
\end{equation}
This condition defines the group $O(2)$, which is the group of all \textbf{orthogonal} 2$\times$2 matrices. It follows that $\det(O)=\pm 1$ -- i.e. the transformations are area-preserving. The subgroup with $\det(O)=1$ is called $SO(2)$, which corresponds to rigid rotations preserving the orientation of the system -- ``S'' denoting \textbf{special}. 

\subsubsection{Rotations with unit complex numbers and $U(1)$}
A unit complex number is a complex number $z$ which satisfies $|z|^2=z^*z=1$. The group $U(1)$ is the set of unit complex numbers, together with ordinary complex number multiplication. The $U$ stands for `\textbf{unitary}`, which generally stands for the condition 
\begin{equation}
U^\dagger U=1,
\end{equation} 
where $U^\dagger = U^{T*}$ is the \textbf{Hermitian conjugate} of $U$. For scalars, the Hermitian conjugate is equivalent to the complex conjugate. Note that a unit complex number can also be denoted as 
\begin{equation}
R_\theta = e^{i\theta} = \cos(\theta) + i \sin(\theta)
\end{equation}
which makes the interpretation of $U(1)$ as rotations on the unit complex numbers evident.

We can connect this description of rotations ($U(1)$) to the previous ($SO(2)$) by defining 
\begin{equation}
1 = \begin{pmatrix}
1 & 0\\
0 & 1
\end{pmatrix}\qquad,\qquad
i = \begin{pmatrix}
0 & -1\\
1 & 0
\end{pmatrix}. \label{eq:map-complex-rot-to-2d-vectors}
\end{equation}
For an arbitrary unit complex number $z = a + ib$, let
\begin{equation}
f(z) = a \begin{pmatrix}
1 & 0\\
0 & 1
\end{pmatrix} + b \begin{pmatrix}
0 & -1\\
1 & 0
\end{pmatrix} = \begin{pmatrix}
a & -b\\
b & a
\end{pmatrix}. \label{eq:isomorphism-so2-u1}
\end{equation}
Since $z = R_\theta = \cos(\theta) + i \sin(\theta)$, we can plug in the real and imaginary components of $z$ into Eq.\eqref{eq:isomorphism-so2-u1} to arrive at Eq.\eqref{eq:2d-rotation-matrix}. We then have $z' = R_\theta z$, to perform rotations. There therefore exists an \textbf{isomorphism} between $SO(2)$ and $U(1)$:
\begin{defn}[Group isomorphism]
Given two groups $(G, *)$, $(H, \astrosun)$, a group isomorphism is a bijective function $f: G \rightarrow H$ such that
\begin{equation}
f(u * v) = f(u) \astrosun f(v) \ \forall\ u, v \in G.
\end{equation}
which is written as
\begin{equation}
(G, *) \cong (H, \astrosun).
\end{equation}
\end{defn}
\noindent $f(z)$ in Eq.\eqref{eq:isomorphism-so2-u1} is therefore a group isomorphism between $U(1)$ and $SO(2)$.


\newpage
\bibliography{physics-from-symmetry.bib} 

\end{document}